\newcommand{\U}[1]{\ensuremath{\,\textrm{#1}}}
\newcommand{\tref}{\ensuremath{\textrm{ref}}}
\newcommand{\vthref}{\ensuremath{v_\textrm{thref}}}
\newcommand{\gkw}{\ensuremath{\texttt{GKW}}}
\newcommand{\gkdb}{\ensuremath{\texttt{GKDB}}}

\documentclass[a4paper]{report}
\usepackage{fullpage}
\usepackage{amsmath}
\usepackage{amssymb}
\usepackage{bm}
\usepackage{graphicx}
\usepackage{tabularx}
\usepackage{hyperref}

\hypersetup{bookmarks, plainpages=false, colorlinks=true, pdfborder={0 0 0 0}, linkcolor=blue, citecolor=blue, urlcolor=blue, pdftitle={GKDB documentation}}

\begin{document}

\title{Documentation of the GKW to GKDB data conversion}

\author{Y. Camenen on behalf of the GKDB working group}

\date{Last update: February 9, 2018}

\maketitle

\chapter{Preamble}
This document describes how to transform inputs and outputs from a GKW flux-tube simulation to match the format used in the GyroKinetic DataBase (GKDB). \\ 
The reader is assumed to have some knowledge of GKW and to have read the documentation of the GKDB.

\tableofcontents

\chapter{Conventions and normalisations}

\section{Coordinate systems}

In GKW, the toroidal direction is defined to have the cylindrical coordinate system $(R,Z,\varphi)$ right-handed whereas in the GKDB it is defined to have $(R,\varphi,Z)$ right-handed, see Fig.\ref{fig:coord1}. In practice, it means that: 
\begin{equation}
\varphi^\gkdb=-\varphi^\gkw
\end{equation}
\begin{figure}[h]
	\begin{center}
		\includegraphics[width=7cm]{GKW_coord.pdf}
		\includegraphics[width=7cm]{GKDB_coord.pdf}
		\caption{\label{fig:coord1} Cylindrical coordinate system used in GKW (left) and the GKDB (right).}
	\end{center}
\end{figure}\\
The flux surface centre definition depends on how the magnetic equilibrium is specified. For \texttt{miller} geometry, the definition of $R_0$ is identical to that used in the GKDB and $Z_0$ is given as an input in the geometry namelist:
\begin{equation}
R_0^\texttt{GKW-miller} = R_0^\gkdb \qquad \qquad Z_0^\texttt{GKW-miller} = \texttt{zmil}R_\tref^\gkw
\end{equation}
For \texttt{chease} geometry, $R_0$ is taken to be the value of \texttt{R0EXP} specified in the \texttt{hamada.dat} file and $Z_0$ is the elevation of the magnetic axis.
\begin{equation}
R_0^\texttt{GKW-chease} = \texttt{R0EXP} \qquad \qquad Z_0^\texttt{GKW-chease} = Z_\textrm{axis}
\end{equation}
The definition of the (dimensional) radial coordinate $r$ is identical in GKW and the GKDB:
\begin{equation}
r^\gkw = r^\gkdb
\end{equation}
The calculation of the  poloidal angle $\theta$ used in the GKDB from GKW inputs is documented in section~\ref{sec:magequil}. At this stage, just notice that most of the time $Z_0^\gkw\neq Z_0^\gkdb$, therefore, the points $s=0$ and $\theta=0$ do not necessarily coincide. 


\section{Reference quantities}
In GKW and the GKDB, all quantities are normalised and made dimensionless by making use of reference quantities. In what follows, normalised quantities are denoted with a "N" subscript. For instance, the normalised version of an arbitrary quantity $\mathcal{A}$ with the dimension of a length will be $\mathcal{A}_N^\gkw=\mathcal{A}/R_\tref^\gkw$ in GKW and $\mathcal{A}_N^\gkdb=\mathcal{A}/R_\tref^\gkdb$ in the GKDB. The conversion from GKW  to the GKDB  involves the ratio of reference quantities. For the example above:
\begin{equation}
\mathcal{A}_N^\gkdb = \frac{R_\tref^\gkw}{R_\tref^\gkdb} \mathcal{A}_N^\gkw
\end{equation}
The ratio of the various reference quantities used in GKW and the GKDB are:
\begin{align*}
q_\textrm{rat} &=  \frac{q_\tref^\gkw}{q_\tref^\gkdb} = - \frac{1}{\texttt{z}^\gkw_{e^-}} & R_\textrm{rat}^\texttt{miller} &= \frac{R_\tref^\texttt{GKW-miller}}{R_\tref^\gkdb} = 1 \\
m_\textrm{rat} &=  \frac{m_\tref^\gkw}{m_\tref^\gkdb} = \frac{m_e}{m_D}\frac{1}{\texttt{mass}^\gkw_{e^-}} & B_\textrm{rat}^\texttt{miller} &= \frac{B_\tref^\texttt{GKW-miller}}{B_\tref^\gkdb} = 1 \\
T_\textrm{rat} &=  \frac{T_\tref^\gkw}{T_\tref^\gkdb} = \frac{1}{\texttt{temp}^\gkw_{e^-}} &R_\textrm{rat}^\texttt{chease} &= \frac{R_\tref^\texttt{GKW-chease}}{R_\tref^\gkdb} = \frac{\texttt{R0EXP}}{R_0^\gkdb}  \\
n_\textrm{rat} &=  \frac{n_\tref^\gkw}{n_\tref^\gkdb} = \frac{1}{\texttt{dens}^\gkw_{e^-}} \frac{n_e(s=0)}{n_e(\theta=0)} & B_\textrm{rat}^\texttt{chease} &= \frac{B_\tref^\texttt{GKW-chease}}{B_\tref^\gkdb} = \frac{\texttt{B0EXP}}{B_0^\gkdb}
\end{align*}
where the $e^-$ subscript denotes the electron species and the electron to deuterium mass ratio is taken to be $\frac{m_e}{m_D}=2.7237 \times 10^{-4}$ in the GKDB.\\
The reference charge, mass, temperature and density ratio can be computed from the electron species parameters in the \texttt{SPECIES} namelist of the GKW input file. The poloidal asymmetry factor for the density can be computed from data in the \texttt{cfdens} file (GKW output).  With \texttt{chease} geometry, the ratios $ L_\textrm{rat}$ and $ B_\textrm{rat}$ can be computed from data in the \texttt{hamada.dat} file (GKW input).\\
The following derived quantities will also be used for the conversion:
\begin{align*}
v_\textrm{thrat} &= \sqrt{\frac{T_\textrm{rat}}{m_\textrm{rat}}} & \rho_\textrm{rat} = \frac{q_\textrm{rat} B_\textrm{rat}}{m_\textrm{rat}v_\textrm{thrat}}
\end{align*}


\chapter{Inputs}
\section{Magnetic equilibrium} \label{sec:magequil}
Only \texttt{miller} and \texttt{chease} magnetic equilibrium specifications are compatible with the GKDB format (\texttt{s-alpha} and \texttt{circ} are not an exact solution of the Grad-Shafranov equation).
\subsection{Flux surface centre}
Let's call $\{R_{\Psi_0},Z_{\Psi_0}\}$ a set of points discretizing the flux surface of interest.\\
The values of $\{R_{\Psi_0},Z_{\Psi_0}\}/R_\tref^\gkw$ are available in the \texttt{geom.dat} file (GKW outpout) and can be used to compute $\{R_0^\gkdb,Z_0^\gkdb\}/R_\tref^\gkw$. 

\subsection{Poloidal angle}
With these values, one can then compute the GKDB poloidal angle $\theta$:
\begin{equation}
 \tan \theta = - \frac{Z_{\Psi_0}/R_\tref^\gkw-Z_0^\gkdb/R_\tref^\gkw}{R_{\Psi_0}/R_\tref^\gkw-R_0^\gkdb/R_\tref^\gkw}
\end{equation}
As the discretisation of the flux surface in \texttt{geom.dat} is done on the GKW $s$ grid, the equation above gives the relationship between $\theta$ and $s$.

\subsection{Radial coordinate}
\begin{equation}
r_N^\gkdb = r_N^\gkw \cdot R_\textrm{rat}= \texttt{eps}\cdot R_\textrm{rat}
\end{equation}

\subsection{Toroidal field and current direction}
\begin{equation}
 s_b^\gkdb = - s_b^\gkw=-\texttt{signb} \qquad \textrm{and} \qquad s_j^\gkw=-s_j^\gkw=-\texttt{signj}
\end{equation}

\subsection{Safety factor}
\begin{equation}
q^\gkdb = s_b^\gkw s_j^\gkw q^\gkw = \texttt{signb}\cdot\texttt{signj}\cdot\texttt{q}
\end{equation}

\subsection{Magnetic shear}
\begin{equation}
\hat{s}^\gkdb = \hat{s}^\gkw = \texttt{shat}
\end{equation}

\subsection{Pressure gradient (entering the curvature drift)}
\begin{equation}
\beta^{' \gkdb}_N = - \beta^{' \gkw}_N \frac{B_\textrm{rat}^2}{R_\textrm{rat}}
\end{equation}
The value of $\beta^{' \gkw}_N $ is taken from \texttt{betaprime\_ref} for \texttt{miller} geometry or from the \texttt{geom.dat} file for \texttt{chease} geometry.

\subsection{Plasma shape}
To compute the shaping Fourier coefficients and their radial derivatives, one first need to compute:
\begin{equation}
 a_N^\gkdb(r,\theta)=\frac{1}{R_\tref^\gkdb}\sqrt{\left[R_{\Psi_0}(r,\theta)-R_0^\gkdb\right]^2 + \left[Z_{\Psi_0}(r,\theta)-Z_0^\gkdb\right]^2} 
\end{equation}
For \texttt{miller} geometry, this can be done by computing $\{R_{\Psi_0}(r,\theta),Z_{\Psi_0}(r,\theta)\}/R_\tref^\gkw$ from the Miller parameters of the GKW input files and then perform the Fourier expansion.\\
For \texttt{chease} geometry $\{R_{\Psi_0}(r,s),Z_{\Psi_0}(r,s)\}$ is directly available in the \texttt{hamada.dat} file and can be used together with the relationship between $\theta$ and $s$ to compute the Fourier coefficients. 

\section{Species}

\subsection{Charge}
\begin{equation}
Z_{sN}^\gkdb = Z_{sN}^\gkw \cdot q_\textrm{rat} = \texttt{z}_s \cdot q_\textrm{rat} 
\end{equation}

\subsection{Mass}
\begin{equation}
m_{sN}^\gkdb = m_{sN}^\gkw \cdot m_\textrm{rat} = \texttt{mass}_s \cdot m_\textrm{rat}
\end{equation}

\subsection{Density}
\begin{equation}
n_{sN}^\gkdb(\theta=0) = n_{sN}^\gkw(s=0) \cdot n_\textrm{rat} = \texttt{dens}_s \cdot n_\textrm{rat} 
\end{equation}
Note than $n_\textrm{rat}$ includes the correction due to poloidal asymmetries ($s=0$ not at the same location as $\theta=0$)

\subsection{Logarithmic density gradient}
\begin{equation}
\frac{R_\tref^\gkdb}{L_{n_s}^\gkdb}(\theta=0) =  \frac{R_\tref^\gkw}{L_{n_s}^\gkw}(\theta=0) \cdot R_\textrm{rat}
\end{equation}
In the presence of poloidal asymmetries, the logarithmic density gradient at $\theta=0$ can be obtained from the \texttt{cfdens} output file. In this file, the first column is the $s$ grid, then comes the logarithmic density gradient for each species and finally the density for each species (all with GKW normalisations). 

\subsection{Temperature}
\begin{equation}
T_{sN}^\gkdb = T_{sN}^\gkw \cdot T_\textrm{rat} = \texttt{temp}_s \cdot T_\textrm{rat}
\end{equation}

\subsection{Logarithmic temperature gradient}
\begin{equation}
\frac{R_\tref^\gkdb}{L_{T_s}^\gkdb} =  \frac{R_\tref^\gkw}{L_{T_s}^\gkw} \cdot R_\textrm{rat} = \texttt{rlt}_s \cdot R_\textrm{rat}
\end{equation}

\subsection{Toroidal velocity}
\begin{equation}
u_N^\gkdb = s_b^\gkw\cdot u_N^\gkw \cdot \frac{v_\textrm{thrat}}{R_\textrm{rat}}  = \texttt{signb} \cdot \texttt{vcor} \cdot \frac{v_\textrm{thrat}}{R_\textrm{rat}}
\end{equation}

\subsection{Toroidal velocity gradient}
\begin{equation}
u^{'\gkdb}_{sN} = s_b^\gkw\cdot  u^{'\gkw}_{sN} \cdot \frac{v_\textrm{thrat}}{R_\textrm{rat}^2} = \texttt{signb} \cdot \texttt{uprim}_s \cdot \frac{v_\textrm{thrat}}{R_\textrm{rat}^2}
\end{equation}

\subsection{Plasma beta}
\begin{equation}
\beta_{eN}^\gkdb = \beta_N^\gkw \cdot n_\textrm{rat} \cdot T_\textrm{rat} \cdot B_\textrm{rat}^2 
\end{equation}
The value of $\beta_N^\gkw$ is either taken from the input \texttt{beta\_ref} of the \texttt{SPCGENERAL} namelist when the  \texttt{beta\_type='ref'} option is used or extracted from the output file \texttt{out} when the  \texttt{beta\_type='eq'} option is used.
 
\subsection{Collisionality}
\begin{equation}
\nu_{eN}^\gkdb = \frac{e^4}{4\pi\varepsilon_0^2}\frac{1}{q_\textrm{rat}^4}\frac{T_\textrm{rat}^2}{n_\textrm{rat}R_\textrm{rat}}\frac{n_\tref^\gkw R_\tref^\gkw}{\left.T_\tref^{\gkw}\right.^2} = \frac{e^4}{4\pi\varepsilon_0^2}\frac{1}{q_\textrm{rat}^4}\frac{T_\textrm{rat}^2}{n_\textrm{rat}R_\textrm{rat}} \frac{\texttt{nref}\cdot\texttt{rref}}{\texttt{tref}^2}
\end{equation}
Note that the expression above assumes that the \texttt{freq\_override} option is set to false. 

\subsection{Debye length}
Always zero in GKW runs.

\chapter{Outputs}

\bibliographystyle{unsrt}
\bibliography{gkw2gkdb}

\end{document}
